\pagestyle{scrheadings}
\chapter*{Epilogue}
\ohead{Epilogue}
\addcontentsline{toc}{chapter}{Epilogue}

Thanks for reading this book to the end. I hope that my delicate efforts do help you in learning \LaTeX{}. However, as Napoleon said, “In war, the moral is to the physical as three is to one.” Similarly, for writing a book, I believe the mentality/spirit of the author is much more important than merely the (\LaTeX{}) techniques or domain knowledge. And so, I want to chip in my personal list of useful resources for equipping authors to all of you as a gift:
\begin{itemize}
    \item \href{https://www.thecreativepenn.com/}{The Creative Penn} (In particular, her book “The Successful Author Mindset”)
    \item \href{https://writershelpingwriters.net/}{Writers Helping Writer ®}
    \item \href{https://janefriedman.com/}{Jane Friedman}
    \item \href{https://www.sfwriter.com/}{Robert J. Sawyer}
\end{itemize}
as well as the masterpiece “Bird by Bird” by Anne Lamott \cite{lamott1995bird}. This is clearly not an exhaustive list and is limited by my exposure, but I think that you will likely find motivations in them. Let me finish with an excerpt taken from “Bird by Bird”:

“I still encourage anyone who feels at all compelled to write to do so. I just try to warn people who hope to get published that publication is not all it is cracked up to be. 

But writing is.

Writing has so much to give, so much to teach, so many surprises. That thing you had to force yourself to do -- the actual act of writing -- turns out to be the best part. It’s like discovering that while you thought you needed the tea ceremony for the caffeine, what you really needed was the tea ceremony. The act of writing turns out to be its own reward.”

Enjoy!

\cleardoubleoddpage
\ohead{Index}
\small
\printindex
\normalsize
\chapter*{Answers to Exercises}
\ohead{Answers to Exercises}
\addcontentsline{toc}{chapter}{Answers to Exercises}
\shipoutAnswer
\cleardoubleoddpage
\ohead{Bibliography}
\printbibliography[heading=bibintoc]