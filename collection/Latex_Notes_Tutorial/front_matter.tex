\begin{titlepage}
\parbox{0.7\textwidth}{\Huge\raggedright\textbf{\textmaybesf{How to Reproduce this Book Exactly with \LaTeX}}}\par
\vspace{2mm}
\parbox[b]{0.9\textwidth}{\large\raggedright\textit{A Self-contained Tutorial on Writing Mathematical Notes}}
\hfill\textcolor{RoyalBlue}{\rule{3mm}{3mm}}\par
\vspace{4mm}\hrule\par
{\Large\raggedleft\textmaybesf{v1.0.0}\hfill C.~L.~Loi\par}
\vfill
{\large\raggedleft A student from \\ 
CUHK-EESC/NTU-AS\par}
\end{titlepage}
\thispagestyle{empty}
\vspace*{\fill}
“How to Reproduce this Book Exactly with \LaTeX”\\
Copyright ©, C.~L.~Loi, 2025. All rights reserved.

\chapter*{Preface}
\addcontentsline{toc}{chapter}{Preface}

The idea of this book was born soon after I completed my first book on Linear Algebra, of course, written using \LaTeX{}. I want to happily share my experiences about working with \LaTeX{} during this long period and help other fellow Mathematics lovers or academics to make their own books. In this process, I have gotten to be more familiar with \LaTeX{} and appreciate both the technical and aesthetic aspects of writing a book. Hence, this book aims to provide a comprehensive tutorial for beginners that covers all the key parts of creating a good-looking Mathematical book/note using \LaTeX{}. The necessary code to exactly produce the outputs is displayed (and hosted on the GitHub repository as well), and there are also a few practice exercises. As the name of this book suggests, readers are encouraged to reproduce the book (no need to really be an exact replica, though; maybe with your own variations instead!) as they progress. This book can also serve as a quick cookbook, and readers can reference any part of it as they see fit. However, notice that this introductory book is by no means exhaustive and cannot substitute for full user manuals. Again, I will be very glad if my work can inspire you to undertake the writing journey of Mathematics.

{\raggedleft Sincerely, \\
Benjamin C. L. Loi 
\par}

(Github Repository: \href{https://github.com/BenjaminGor/Latex_Notes_Tutorial}{https://github.com/BenjaminGor/Latex\_Notes\_Tutorial})

\cleardoubleoddpage
\tableofcontents